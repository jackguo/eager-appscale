The growth of the World Wide Web (WWW), web services technology and cloud computing has significantly influenced the
way developers design and implement software applications. Instead of implementing all the required functionality from
the scratch or using standalone libraries, developers are content to offload as much functionality as possible to remote,
web-accessible application programming interfaces (web APIs) hosted in the cloud, thereby significantly reducing the 
programming workload. This also allows developers to indirectly benefit from the efforts put in by the API and cloud providers to 
increase the reliability, scalability and availability of their respective service offerings. 

On the other hand, this phenomenon also motivates development teams around the globe to expose their applications to 
the WWW as APIs, so that other developers can easily reuse existing software packages, and construct powerful new systems 
based on cloud-hosted web APIs. As a result, today, we are experiencing a proliferation of web APIs. According to the 
statistics published by the popular web API index, ProgrammableWeb, the number of publicly available
web APIs has shown nearly 100\% annual growth rate, from 1000 APIs in 2008 to 11000 APIs in 2013. This is a significant upshot 
considering the fact that it took almost 3 years for ProgrammableWeb to accumulate that initial 1000 APIs. 

What's even more interesting is that this uprising 
of web APIs is not just contained to the IT industry. In fact, all industries that use the WWW as a medium for conducting their 
business have shown strong interest in exposing their digital assets as web APIs. Consequently, as of March 2014, ProgrammableWeb
lists 256 APIs for advertising, 355 APIs for shopping, 227 APIs for travel and 224 APIs for education. Also,
in the recent years some non-commercial entities have started publishing web APIs. APIs provided by bodies like IEEE, UC Berkeley and 
US White House are good examples in this category. As far as the government sector is concerned, the US government memorandum 
on ``Building a 21st Century Digital Government'' issued in 2012 directs all federal agencies to focus their IT strategy around creating
public and open web APIs hosted in the cloud.

As more and more web APIs are being deployed in the cloud, the need for effectively managing and governing these artifacts also
continues to grow. Without proper governance, API providers set themselves up for potential security breaches, denial of service (DoS)
attacks, poor code reuse, violation of promised service-level agreements (SLAs), naming or branding issues and abuse of key digital 
assets by the API consumers. Unfortunately, most existing cloud platforms, especially platform-as-a-service (PaaS) offerings that enable
deploying web APIs, only facilitate ungoverned API deployments. In other words they only support implementing the business logic of
web services (database access, task scheduling, HTTP request processing etc.), and put the onus of API versioning,  access control, 
dependency management and policy enforcement to the developers who create and deploy the web APIs. A number of so called
``API management'' solutions (e.g. 3Scale, Apigee, Layer7) have emerged to fill this gap. These third party solutions offer access control
via API keys (typically based on OAuth), access rate limiting and several other useful API governance features. But such third party 
solutions have several limitations when it comes to governing APIs hosted in the cloud.

\begin{enumerate}
\item Third party API management solutions operate as services external to the target cloud that actually hosts the APIs. Therefore 
the API providers have to manage and potentially pay for an additional service, which increases the management overhead and overall cost.
\item Since the API management solutions are external to the target cloud, the ability to control and govern web APIs in an enforced manner is
lost. In other words, it is not possible to perform crucial design-time and deployment-time validations and governance checks for the APIs
that are being developed and rolled out into the cloud. 
\item The API management solutions fail independently of the target cloud, thereby affecting the reliability, scalability and availability of the 
cloud-hosted APIs. This implies that the API providers cannot take advantage of the potential scalability and availability benefits of using a cloud
platform as a deployment target.
\end{enumerate}

We postulate that a cloud platform can facilitate effective and enforced governance in a developer-friendly and cost effective manner by
making API governance and the related tooling an integrated component of the cloud itself. That is, instead of using a third party API management
solution that simply layers governance features on top of the cloud, we propose that cloud platforms should provide API governance as a 
core service which is always available for the developers that create web APIs. In this regard, we view enforced API governance as a 
fundamental feature of the cloud similar to other key traits of clouds such as fault-tolerance and auto-scaling. With this
approach, API providers do not have to manage any external governance services or configure any API management solutions. They simply
specify the governance policies that must be enforced by the target cloud, and from there onwards all the web APIs deployed on the cloud
are subjected to proper governance automatically.

Based on this philosophy, we propose EAGER (Enforced API Governance Engine for REST), a model and an architecture that augment existing
PaaS clouds for facilitating API governance as a cloud-native feature. EAGER enforces proper versioning of APIs and supports dependency 
management, access control and comprehensive policy enforcement at the API deployment time. By focusing our efforts on deployment-time
API governance, we strive to ensure that APIs rolled out into production clouds always meet the required developer and other stakeholder
expectations, and they are ready to be consumed by potentially thousands of users over long periods of time. Our dependency management
and policy enforcement support ensures that developers always reuse existing APIs as much as possible when creating new software artifacts,
while staying away from unverified local or third party dependencies. EAGER also tracks changes made to already deployed web APIs and prevents
any backwards incompatible API changes from being rolled out into production, thereby preventing downstream applications from breaking.

We further enhance our approach with a unified language for specifying all kinds of
API governance policies. Our language deviates from today's existing policy languages like WS-Policy and WS-Agreement (which are 
primarily XML-based), and uses developer-friendly Python syntax for specifying even the most complex policy statements in a simple and 
intuitive manner. Also we ensure that specifying the required policies is the only additional activity that API provides should perform in
order to benefit from EAGER. All other API governance related verification and enforcement work is carried out by the cloud platform fully
automatically.

To evaluate the feasibility and performance of the proposed architecture, we implement EAGER as an extension to AppScale, an open source
PaaS that mimics Google App Engine. We show that EAGER architecture can be easily implemented in today's PaaS clouds with
minimal changes to existing cloud platforms. We further show that EAGER API governance and policy enforcement incur a negligibly small 
overhead to the API deployment process. For applications that do not export any web APIs the overhead is barely noticeable. 
When deploying applications that export multiple web APIs, the governance overhead grows linearly in the number of APIs being exported 
by the application. Finally, considering the rapid growth rate of web APIs in the cloud, we show that EAGER scales well to tens of thousands of
deployed web APIs and hundreds of user defined governance policies.


